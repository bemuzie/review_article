\chapter{Обзор литературы} \label{rewiev}
\section{Лучевая диагностика рака поджелудочной железы} \label{pancreas_radiology}
\subsection{Выявление}
\subsubsection{СКТ}
Высокие операционные характеристики теста, простота выполнения и доступность сделали СКТ наиболее используемым методом выявления РПЖ. 
Визуализация РПЖ основана на выявлении разницы денситометрической плотности опухоли и окружающей ткани, создаваемого масс-эффекта, пакреатической и/или билиарной гипертензией. 
Различия в васкуляризации опухоли и паренхимы позволяют визуализировать опухоль как гиподенсное образование на изображениях, полученных после контрастного усиления. Поэтому важно проводить сканирование в фазу наибольшего усиления паренхимы ПЖ - примерно 40 с после введения контрастного вещества. Исследования, проведенные в панкреатическую фазу, обладают на 24\% большей чувствительностью по сравнению с исследованиями, проведенными в портальную и на 35\% в артериальную фазу [5]. Рекомендуется выполнять исследование до установки стента, в связи с тем, что наличие стента вызывает появление артефактов [6]. Данных о степени их влияния на диагностическую эффективность в доступной литературе не найдено.

В целом чувствительность КТ в выявлении РПЖ составляет 75–100\% , специфичность 70–100\% [7–14]. Выявление образований диаметром выше 2 см не представляет проблем для КТ, чувствительность метода в данном случае превышает 98\%. [10]
Чувствительность СКТ в выявлении опухолей диаметром  до 20~мм лежит между 18 и 78\%. Результаты определения чувствительности СКТ в выявлении мелких опухолей ПЖ представлены в таблице 1.

Таблица 1. Чувствительность СКТ в выявлении мелких опухолей ПЖ
Исследование	Размеры (мм)	Чувствительность, (выявлено/число больных)
Legmann et al., 1998	<15	67\% (4/6)
Bronstein et al., 2004	<20	78\% (14/18)
Ichikawa et al., 1997	<20	58\% (7/12)
Rose et al., 1999	<20	18\%(-)

\begin{table}[htbp]
        \caption{Чувствительность СКТ в выявлении мелких опухолей ПЖ}
        \begin{flushleft}
        \begin{tabular}{|p{6cm}|c|c|c|}
        \hline
        Исследование & Размеры (мм) & Чувствительность & (выявлено/число больных)\\ \hline
        Legmann et al., 1998 & $<15$ & 67\% & (4/6) \\ \hline
        Bronstein et al., 2004 & $<20$ &  78\% & (14/18) \\ \hline
        Ichikawa et al., 1997 & $<20$ &  58\% & (8/12) \\ \hline
        Rose et al., 1999 & $<20$ &  18\% & (---) \\ \hline
        \end{tabular}
        \end{flushleft}
        \label{ct_sensitivity}
\end{table}

Около 5-10\% опухолей ПЖ по данным визуальной оценки изоденсны при контрастном усилении [15,16], не смотря на то что денситометрическая плотность опухоли и паренхимы ПЖ может отличатся. В исследовании Prokesh et al. разница в денситометрической плотности между “изоденсной” опухолью и паренхимой составила $9.25\pm11.3~HU$ в панкреатическую фазу и $4.15\pm8.5~HU$ в портальную. Для гиподенсных опухолей разница была $74.76\pm35.61~HU$. Считается, что для визуализации образования разность денситометрической плотности между опухолью и паренхимой должна составлять не менее 10 HU [17]. Следует принимать во внимание, что в связи с отсутствием когортных исследований, посвященных скринигу РПЖ с помощью всего спектра методов лучевой диагностики, оценить реальную распространенность опухолей такого рода невозможно. Средний размер изоденсивных опухолей составляет 30~мм (от 15 до 40~мм). Иногда такие опухоли не удается визуализировать на макропрепарате [8]. Гистологически изоденсные раки ПЖ характеризуются более низкой клеточной плотностью, большим числом ацинусов и меньшим некрозов, что расценивается как ранние изменения при раке ПЖ [18]. Заподозрить наличие опухоли в таких случаях можно по косвенным признакам: массэффекту, обрыву и расширению панкреатического протока, атрофии паренхимы ПЖ дистальнее места обрыва протока. Прямая визуализация таких опухолей часто возможна с помощью МРТ с контрастным усилением (Чувствительность: 79,2\%) или ПЭТ с 18F-ФДГ (Чувствительность: 73,7\%)[19]. Проблема выявления новообразований такого рода особенно актуальна, т.к. они, как правило, резектабильны и выживаемость в этой группе пациентов выше [18].
\subsubsection{МРТ}
\subsubsection{ПЭТ}
\cite{Milojevic2013}
\subsubsection{УЗИ}
\subsubsection{ПКТ}
\subsection{Дифференциальный диагноз}
\subsubsection{СКТ}
\subsubsection{МРТ}
\subsubsection{ПЭТ}
\subsubsection{УЗИ}
\subsubsection{ПКТ}
\subsection{Оценка распространённости}
\subsubsection{СКТ}
\subsubsection{МРТ}
\subsubsection{ПЭТ}
\subsubsection{УЗИ}
\subsubsection{ПКТ}

СКТ

\subsection{Магнитно зезонанстная томография}
МРТ

По данным многих авторов МРТ не имеет преимуществ перед КТ в выявлении солидных опухолей поджелудочной железы [21]. Диагностическая точность МРТ составляет 70\% [20].  Известно, что качество визуализации вирсунгова протока выше при МРТ, что должно облегчить визуализацию косвенных симптомов опухоли поджелудочной железы.
Для первичной диагностики МРТ применяется при невозможности выполнения КТ или если результаты КТ неубедительны [22].

эндоУЗИ

Одним из наиболее чувствительных методик в выявлении РПЖ является эндоУЗИ. Чувствительность метода близка к 100\%, даже в отношении образований диаметром менее 2см [23]. 
Однако относительная инвазивность, сложность выполнения, операторозависимость не позволяют использовать метод в качестве метода “первого рубежа”. По мнению большинства авторов к эндоУЗИ следует прибегать только при сомнительных результатах КТ и МРТ [24].

ПЭТ с 18F-ФДГ

Рак ПЖ визуализируется как очаг накопления 18F-ФДГ. SUV составляет 2-10 и зависит от многих факторов (размер опухоли, стадия, концентрация глюкозы в крови). Визуализация может быть значительно затруднена при малых размерах опухоли и ее расположении вблизи участков высокого физиологического накопления 18F-ФДГ. Примером такого рода опухолей являются ампулярные опухоли, имеющие, как правило, малые размеры и расположенные вблизи стенки кишки [25].

В большинстве случаев ПЭТ превосходит другие методы в диагностике РПЖ. По данным Inokuma et al [26] чувствительность и специфичность ПЭТ составляют (96\% и 78\%), что выше, чем при КТ (91\% и 56\%), и эндоУЗИ (96\% и 67\%).

Данные о диагностике мелких опухолей ПЖ с помощью ПЭТ противоречивы. Rose et al. указывают, что для образований диаметром менее 2 см чувствительность КТ составляет 18\% против 100\% при ПЭТ. Но КТ проводилась указанными авторами с толщиной среза 5 мм, что повлияло на чувствительность исследования. Вместе с тем, низкое пространственное разрешение и невозможность точной локализации очагов накопления не позволяют использовать ПЭТ в качестве самостоятельного диагностического метода [25,26].

Чувствительность и специфичность ПЭТ с 18F-ФДГ в выявлении РПЖ составляет 46-71\% и 63-100\%, соответственно [18] 
Можно выделить две основных причины ложноотрицательных результатов при ПЭТ: гипергликемия и ранние стадии рака ПЖ. Наиболее частой причиной ложноотрицательных результатов при ПЭТ является повышенный уровень глюкозы в крови [27]. Чувствительность ПЭТ у пациентов с гипергликемией падает до 63-42\% [28][27,29,30]. 
У пациентов с нормальным уровнем глюкозы при ранних стадиях рака ПЖ (I-II) в 3\% случаев наблюдаются ложноотрицательные результаты [31].

Эффективность визуализирующих методов в дифференциальной диагностике рака поджелудочной железы

Дифференциальный диагноз гиподенсных образований поджелудочной железы в основном проводится между очаговыми формами панкреатита (хронического псевдотуморозного и аутоимунного (АИП)) и РПЖ. Примерно 5-10\% проводимых резекций поджелудочной железы по поводу злокачественных опухолей при гистологическом исследовании оказываются доброкачественными образованиями и примерно четверть из них АИП. [32,33]. Различить оба патологических состояния трудно не только радиологам, но и гистологам, так как они могут иметь общие черты [34]. У пациентов с хроническим панкреатитом в 15-20\% результаты биопсии являются ложно отрицательными [23,35].
Воспалительные явления могут сопутствовать аденокарциноме, опухоль может развиваться на фоне хронического панкреатита. У пациентов с 10 летним анамнезом панкреатита риск развития РПЖ составляет 2\%, у лиц, болеющих более 20 лет риск увеличивается до 5-6\% [36,37].
Оба процесса отображаются как гипоэхогенные при УЗИ, гиподенсные при КТ, и имеют одинаковую интенсивность сигнала на T1- и T2- взвешенных МРТ изображениях. Симптом «двухстволки», стриктуры протоков, инфильтрация парапанкреатической клетчатки, вовлеченность артерий и перипанкреатическая венозная обструкция могут присутствовать при обеих патологиях [38].
Симптомами, характерными для панкреатита, являются умеренная атрофия тела железы, постепенное сужение вирсунгова протока, визуализация не расширенных протоков, проходящих сквозь опухоль, (“duct penetrating sign” [38]), неровность контуров протока, кальцификаты в поджелудочной железе. Для РПЖ характерна резкая обструкция протока поджелудочной железы с атрофией ее паренхимы дистальнее места обструкции [34].
Очаговые формы панкреатита гораздо сложнее дифференцируются с аденокарциномой. Очаги визуализируются как гипо-/изоденсные образования при КТ и гипоинтенсивные образования при МРТ [39–42]. Из-за высокой степени накопления 18F-ФДГ, дифференциальный диагноз с помощью ПЭТ также невозможен [43]. Ведущую роль в этой клинической ситуации играют методы лабораторной диагностики, в частности определение уровня IgG и IgG4. Однако, Kim et al. отмечают, что иногда (8\% 1/12) они могут быть повышены у пациентов с изоденсным раком ПЖ [18]. Недостаточное число публикаций по этой тематике не позволяет сделать выводы о распространенности проблемы.
Характерным признаком АИП считается стеноз сегмента Вирсунгова протока протяженностью свыше 3 см, при диаметре супрастенотической части Вирсунгова протока не выше 6 мм [42]. Ни у кого из 7 пациентов с АИП при РХПГ не было визуализировано расширенного до 5 мм вирсунгова протока с сопутствующей атрофией паренхимы поджелудочной железы дистальнее места обструкции. 
Данные литературы о роли динамики накопления контрастного вещества в дифференциальном диагнозе между РПЖ и панкреатитом противоречивы. Ранние исследования показали невозможность отличить оба процесса на основании динамики усиления [44,45]. Действительно, гиповаскулярные образования поджелудочной железы не имеют четкого паттерна контрастирования, позволяющего их дифференцировать. Вместе с тем, для некоторых образований выделение таких паттернов возможно. Так, у 85,7\%(6 из 7) пациентов с очаговой формой АИП в панкреатическую фазу определялось гиподенсное образование, изоденсное в портальную фазу [44].
Более поздние исследования, напротив, продемонстрировали потенциальную возможность дифференциального диагноза на основании динамики накопления КВ показали, что анализ кривых «интенсивность сигнала/время» является полезным подходом, т.к. аденокарцинома имеет более поздний пик накопления по сравнению с псевдотуморозным панкреатитом (медленный подъем до пика на 180 секунде был характерен для аденокарцином) [45]. Kishimoto et al. продемонстрировали возможность дифференциального диагноза между двумя патологиями на основании анализа кривых «концентрация/время» при КТ. При псевдотуморозном панкреатите авторы определяли отсроченное вымывание КВ, в то время как при РПЖ наблюдали постепенное увеличение плотности с пиком после 150 секунды [46].
Диффузионно взвешенные МРТ и ПЭТ с 18F-ФДГ также являются многообещающими подходами в дифференциальном диагнозе. Опухоли, для которых характерно снижение диффузии в связи с высокой плотностью клеток, имеют сигнал высокой интенсивности и более низкий кажущийся коэффициент диффузии (ККД) по сравнению с нормальной тканью. В исследовании диффузионно взвешенной МРТ 38 пациентов, Fattahi et al обнаружили что при значении «b» 600 sec/mm2, псевдотуморозный панкреатит не был отличим от ткани поджелудочной железы, в то время как аденокарциномы имели гиперинтенсивный сигнал. Средний ККД для аденокарциномы ($1.46\pm0.18 *10-3 mm^2/sec$) был значимо ниже, чем его значение для псевдотуморозного панкреатита ($2.09\pm0.18* 10-3 mm^2/sec$) или нормальной паренхимы ($1.78\pm0.07 * 10-3 mm^2/sec$) [47].
Вместе с тем, диффузионно взвешенные изображения не рационально использовать для прямой визуализации образований в связи с тем, что примерно в половине случаев опухоль и дистальная часть паренхимы имеют гиперинтенсивный сигнал [48].
Чувствительность и специфичность эндоУЗИ в отношении дифференциального диагноза РПЖ и панкреатита составляет 85\% и 60\%. В проспективном исследовании Varadarajulu et al. показали, что в группе пациентов перенесших ГПДР с диагностированным при гистопатологическом исследовании хроническим панкреатитом, эндоУЗИ позволило добиться 90.5\% чувствительности, 85.7\% специфичности и 88.1\% точности, в то время как КТ не позволила провести верный дифференциальный диагноз ни в одном случае [32]. 
Помимо этого эндоУЗИ позволяет провести аспирационную биопсию с последующим цитологическим исследованием, что повышает диагностическую ценность исследования (Чувствительность {~}$80\pm90\%$, Специфичность {~}$95\pm100\%$) [49–54].

Несмотря на свои преимущества, биопсия не имеет абсолютной чувствительности. Доля ложно отрицательных результатов составляет 15-20\%. Что возникает по причине сопутствующего панкреатита или фиброза, неверного выбора места биопсии, аспирации крови, ошибочной интерпретации полученного материала [23,52].

ПЭТ 

Среди функциональных подходов наиболее эффективным является ПЭТ. Ряд исследований показали превосходство ПЭТ в оценке природы гиповаскулярных образований ПЖ [55–57].
Так как и злокачественные опухоли, и очаги воспаления обладают высоким метаболизмом глюкозы, различить их с помощью ПЭТ достаточно трудно при любой локализации изменений. Дифференциальный диагноз основан на оценке уровня SUV через 1 и 2 часа после введения РФП.

В исследовании 80 пациентов Friess et al. обнаружили увеличение накопления 18F-ФДГ у 41 из 42 пациентов с аденокарциномой и отсутствие увеличения захвата у 28 из 32 (88\%) пациентов с хроническим панкреатитом [31]. Однако, следует иметь в виду, что при средних значениях SUV (от 3.37 до 7.7 через 1~ч и от 3.53 до 9.98 через 2~ч) дифференциальный диагноз не возможен. Относительно уверенно РПЖ можно диагностировать если SUVmax выше указанного интервала [58].
Ложно положительные результаты могут наблюдаться у пациентов с ретроперитонеальным фиброзом, разделенной поджелудочной железой и с сопутствующим панкреатитом [31], после установки назобилиарного катетера, тромбоза портальной вены, кровоизлияния в псевдокисту ПЖ [59], постлучевого панкреатита, зонах недавнего оперативного вмешательства или биопсии, стентирования холедоха [25]. В описанных случаях дифференциальный диагноз может быть выполнен с помощью других подходов. 
Вопрос полезности оценки накопления 18F-ФДГ в динамике для дифференциального диагноза остается открытым. Большинство исследований демонстрируют увеличение SUV в период от 1 до 2 часов после инъекции 18F-ФДГ у больных раком поджелудочной железы, в то время как в доброкачественных образованиях SUV уменьшается [60]. Однако ряд исследователей не смогли найти такой зависимости [58].
К недостаткам ПЭТ можно отнести более высокую лучевую нагрузку, длительное время сканирования, что снижает качество исследования за счет движения больного. Кроме того, применение алгоритмов коррекции аттенуации снижает точность и воспроизводимость при вычислении стандартизированного уровня захвата [30].
Таким образом, диагностика рака ПЖ на ранних стадия проблематична как для функциональных, так и для морфологических, и вряд ли может быть решена увеличением их разрешающей способности. 
Несмотря на значительный прогресс в методике выполнения исследований и интерпретации их результатов, существенный успех достигнут лишь для крупных опухолей. Выявление и дифференциальная диагностика мелких образований поджелудочной железы трудны, требуют мультимодального подхода. Для многих наблюдений остаются справедливыми выводы первых исследований диагностики злокачественных образований ПЖ говорящие о невозможности дифференциального диагноза [27,61].

\section{Анализ перфузии средствами лучевой диагностики} 
\label{perfusion_radiology}
Интерес к изучению перфузии ткани зародился в %давно 
Методы изучения отличались высокой инвазивностью и низкой точностью. Математические модели микроциркуляции были примитивны, поэтому количественная оценка каких-либо физиологических показателей была невозможна. Однако проведенные исследования позволили получить базовое представление об организации и регуляции микроциркуляторного русла. 
Развитие технологий 
Микроциркуляторное русло сложная динамическая структура. Дело в том что если заполнить капилляры какой-либо затвердевающей смесью получившаяся in vitro сосудистая сеть будет довольно сильно отличаться от in vivo варианта, т.к. примерно …% капилляров находятся в закрытом состоянии [], а диаметр капилляра в нормальных условиях около … % от максимального. Собственно «усиление микроциркуляции» происходит в основном за счет их открытия  и в незначительной степени за счет их расширения[].  Более того, количественный анализ такого слепка весьма не тривиальная задача. На рис.1 представлена фотография такого слепка. На первый взгляд кажется, что можно точно измерить относительный объем сосудистого русла в ткани, но 
По сути, мы пытаемся ответить на вопрос: «Насколько выражена локальная микроциркуляция ?»
Степень выраженности микроциркуляции в зоне интереса оценивается по следующим параметрам:
Скорость диффузии ионов.
Объемная скорость крови
Объем крови
Время транзита крови
Относительное число капилляров


Довольно широкий диапазон заболеваний начинается с невыраженных и довольно локальных морфологических изменений. Увидеть их можно, во-первых, с помощью методов с высоким пространственным разрешением (КТ, МРТ, гистологические методы) или с помощью методов «функциональной» визуализации, к которым относятся ПЭТ, сцинтиграфия, МРТ-спектроскопия, МРТ «диффузия». 

Пока что, прямая визуализация микроциркуляторного русла возможно только путем анализа микропрепарата. ….. имунногистохимические методы с  применением эндотелий специфичных маркеров.


Второй подход основан на том, изменения структуры влечет изменения функции . 

этот факт является причиной непрерывной гонке за пространственным разрешением 


Основной целью перфузионных исследований является оценка состояния микроциркуляции. Однако, такая простота определения весьма обманчива, т.к. параметров характеризующих микроциркуляцию довольно много. «Микроциркуляция» 
